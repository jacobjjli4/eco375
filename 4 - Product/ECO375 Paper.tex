\documentclass{eco_375_paper}
\addbibresource{\opath/citations.bib}
\begin{document}

% Title page
\vspace*{\fill}
\begin{center}
	How do 1930s HOLC redlining maps affect modern political outcomes?
	
	By: Benjamin Lee (1007236475) \& Jacob Li (1006824750)
	
	benjaminsw.lee@mail.utoronto.ca \& jacobjj.li@mail.utoronto.ca
	\bigskip
	
	ECO375: Applied Econometrics 
	
	University of Toronto
	
	Department of Economics
	
	March 28, 2023
	
	\bigskip
	\textbf{Abstract: }
	
	How do discriminatory mortgage lending practices based on maps drawn in the 1930s impact modern-day political outcomes? We investigate how mortgage lending desirability grades assigned by the Home Owners Loan Corporation (HOLC) in the 1930s affects people’s voting behaviour in the 2020 United States general election. Using new 2020 elections data at the census tract level, we use a multiple linear regression framework to estimate how the proportion of a census tract graded C or D affects electoral outcomes in that tract while controlling for race, education, age, income, gender, and city fixed effects. We precisely estimate that HOLC map boundaries have little to no association with modern political outcomes, in contrast with the significant and long-lasting socioeconomic consequences of HOLC maps. Our results, however, cannot be deemed causal due to the potential for omitted variables bias and measurement error and requires additional external validation to be generalized.
\end{center}
\vspace*{\fill}
\pagebreak

% Body
\section*{Introduction}
In 1933, the Home Owners Loan Corporation (HOLC) was created in the wake of widespread foreclosures during the Great Depression. One of the HOLC’s primary objectives was to assess lending risk in major cities, which meant grading neighborhoods on a scale of A to D based on a variety of characteristics, many of which were discriminatory. For instance, neighborhoods with a greater number of poor households, immigrants, and non-White residents were assigned lower grades. These findings were summarized on “Residential Security Maps,” which would go on to influence mortgage insurance issuance and private lending behavior, barring access to neighborhoods with greater resources. This type of discrimination is referred to as redlining. The cities in which this redlining occurred are mapped in Figure 1.

Current literature on redlining suggests that it has had a substantial long term impact on a host of outcomes for residents (Swope et al., 2022). In terms of socioeconomic outcomes and economic opportunity, Aaronson et al. (2021) found that redlining maps had economically significant causal effects on the impact of neighborhoods on labor market outcomes, family structure, and incarceration, where residents who grew up on the lower-graded side of neighborhood borders faced negative effects, even decades after the maps were created. Moreover, there is a large body of existing research on the impact of redlining on a variety of health-related outcomes, including life expectancy, COVID-19 risk factors (Richardson et al., 2020), asthma (Nardone et al., 2020; Schuyler and Wenzel, 2022), and modern gun violence (Benns et al., 2020).
\section*{Context and Data}
Our baseline model uses neighborhood level data on redlining from the University of Idaho’s Mapping Inequality project (Nelson et al., 2020) and census block group-level 2020 election data from the Harvard Dataverse (Bryan, 2022). We note that South Dakota, Kentucky, and West Virginia did not report their voting data. 

To generate our dataset, we start by amalgamating individual census block group-level observations to the census tract level. We do this by adding all the votes for Democrats, Republicans, Libertarians, and others for each census block group within a specific census tract. This is easy to do since each 12-digit block group code shares its first 11 digits with other block groups of the same tract.

We then merge this with the second dataset on the census tract. We use a many-to-one merge since one observation in the redlining data will correspond to many observations in the voting data, since observations in the former are at the HOLC neighborhood level. We then calculate the proportion of each census tract that is redlined, or assigned a Grade of D using the proportion of the census tract within a specific neighborhood, which is given in the redlining data.

We also add covariates capturing a range of features related to political outcomes, which include income, racial demographics, age, and education levels. These features have been shown to be related to political outcomes and party affiliation. For instance, Democrats tend to have lower incomes, are younger and more racially diverse, and pursue more advanced degrees than their Republican counterparts (Fay, 2021). We use census tract level data from the American Community Survey (ACS) for these additional covariates.

After the initial merge of the redlining and voting data, we are left with 16,363 census tract level observations. For the purposes of our analysis, we only keep census tracts of which more than 90 percent has been mapped by the HOLC, and are thus left with 6,116 observations, which further declines to 6,026 observations after adding covariates due to missing values.

Summary statistics for our data are found in Table 1. The Grade A-D Share variables measure the proportion of each tract that lies in a neighborhood that has been assigned that corresponding grade level. Grade D Share is our main independent variable, and is meant to capture the level of redlining in each observation. Based on the mean values, the average census tract in our dataset would consist mostly of neighborhoods with a grade of C, and least of neighborhoods with a grade of A. The average proportion of lower-graded neighborhoods is greater than that of higher-graded neighborhoods, suggesting that most census tracts had a relatively high amount of redlining.

Our main outcome variable and metric for political outcomes is Democratic share, which captures the proportion of total votes cast in each census tract that were for the Democratic party. Notably, the mean is 0.791, meaning that the average census tract among those analyzed voted in favor of Democrats by a relatively substantial margin.

The remaining variables capture the additional covariates used in our extension results, including median income, median age, the male-to-female ratio, education levels, and racial demographics of the voting age population. 

${\textit{dem\_share}_i} = {\beta_0} + {\beta_1}{\textit{d\_share}_i} + {\beta_2}{\textit{c\_share}_i} + {\beta_3}{\textit{d\_share}_i^2} + {\beta_4}{\textit{d\_share}_i^3} + {\beta_5}{\textit{c\_share}_i^2} + {\beta_6}{\textit{c\_share}_i^3} + \boldsymbol{{\gamma}} \boldsymbol{X_i} + u_i$
\section*{Regression Analysis}
We begin with a simple linear regression of Democratic share (dem\_share, representing political outcomes) on Grade D share (d\_share, representing the level of redlining) (Table \ref{table:main_regressions}, specification (1)) modeled by $${\textit{dem\_share}_i} = {\beta_0} + {\beta_1}{\textit{d\_share}_i} + u_i$$. We use robust standard errors to account for heteroskedasticity. Our estimated coefficient for $\beta_1$ is 0.0739, meaning that a 1 percentage point increase in a census tract’s Grade D area is associated with 0.0739 percentage point increase in Democrat vote share. With a t-statistic of 17.85, this association is statistically significant. We therefore reject the null hypothesis that $\beta_1 = 0$ that there is no association between redlining and political outcomes. Our estimated effect is economically insignificant, since a 40.1 percentage point increase in Grade D share (a one standard deviation change) only leads to a 2.96 percentage point increase in Democratic share, which is approximately one fifth of a standard deviation in Democratic share.

One major limitation of our simple linear regression is omitted variable bias. We know that redlined neighborhoods are associated with having more African Americans, poverty, and lower levels of education, which are in turn associated with political outcomes. As such, our estimated coefficient on redlining may capture other effects that are correlated with redlining, leading to bias in our result. We cannot even tell which direction our bias goes – redlining is correlated with African American share, and African Americans are more likely to vote Democrat than the national average. Redlining is also correlated with lower education levels, and those with lower education levels are less likely to vote Democrat than the national average.

Using Grade D share as our measure of redlining may not accurately measure the level of redlining in a census tract. Our simple model treats two tracts with the same Grade D area as having the same level of redlining regardless of how the rest of the tract is graded. For example, the simple model considers a tract that is half Grade D and half Grade A to have the “same” level of redlining as a tract that is half Grade D and half Grade C. Whether a neighbourhood was graded C, or “Declining”,  may also constitute redlining, although to a lesser degree than neighbourhoods graded D.

Our model also assumes a linear relationship between redlining and political outcomes. The true marginal effect of redlining on political outcomes may vary depending on the current level of redlining, or it might vary depending on the demographics of the tract. Finally, our model fails to account for spatial error clustering. Since our observations are distributed over 166 cities, there may exist significant intra-city correlation. Our heteroskedasticity robust standard errors may therefore be too small, leading to inflated significance of our results.
\section*{Multiple Linear Regression}
For the remaining specifications (2) to (6), we add city fixed effects and cluster standard errors at the city level. City fixed effects allow us to control for unobserved determinants of political outcomes at the city level. Some cities, for example, may be more likely to vote Democratic as a whole, regardless of a specific census tract’s level of redlining. Clustering standard errors at the city level allows us to control for intra-city correlation. Using robust standard errors would inflate the significance of our estimates since they treat our observations as completely independent, when in reality, there may be a structural determinant of the error term that depends on the city.

Specification (2) corrects for our measurement of redlining by including Grade C share as a covariate. Our estimate for the coefficient on Grade D share is 0.0804 and remains statistically significant. Our estimate on Grade C share is 0.0222 with a standard error of 0.0114, which is statistically significant at the 10\% level. The estimate, however, is economically insignificant since a 1 percentage point increase in a tract’s Grade C share is associated with a 0.0222 percentage point increase in a tract’s Democratic vote share. This result is nonetheless consistent with our finding that redlining, as measured by Grade D share, increases Democrat vote share.

Specification (3) corrects for non-linearities in the relationship between redlining and political outcomes by including quadratic and cubic terms on Grade C share and Grade D share. Given the significance of the quadratic and cubic terms on Grade C share and Grade D share, it appears that the actual relationship between redlining and political outcomes is highly non-linear. Since our model is cubic in Grade D share, we notice that, holding all else equal, the relationship between Grade D share and Democrat vote share is increasing between 0 and 29.6\% Grade D share, decreasing between 29.6\% and 72.3\% Grade D share, and increasing again between 72.3\% and 100\% Grade D share. Similarly, the relationship between Grade C share and Democrat vote share is increasing between 0 and 22.2\% Grade C share, decreasing between 22.2\% to 71.9\% Grade C share, and increasing again from 71.9\% Grade C share to 100\% Grade C share. We can therefore say that additional redlining is associated with an increase in Democrat vote share when redlining is low or high and a decrease in Democrat vote share when redlining is medium.

Specification (4) corrects for omitted variable bias by including covariates on census tract-level demographics such as income, male-female ratio, educational attainment, and race, all of which are known to affect political outcomes. Compared to Specification (2), the coefficient on Grade D share has dropped from 0.0804 to 0.0312 while the coefficient on Grade C share has dropped from 0.0222 to 0.0154, indicating that Specification (2) suffered from omitted variables bias. In particular, one of the covariates we added in Specification (4) may have been positively correlated with redlining and with Democrat vote share or negatively correlated with redlining and with Democrat vote share, creating positive omitted variables bias. Furthermore, the introduction of controls has resulted in our coefficients on Grade C share and Grade D share to become statistically insignificant. That is, after controlling for variation in important tract-level demographics, the residual variance in Democrat vote share, Grade C share and Grade D share is not enough to precisely estimate the effect of redlining on political outcomes. We therefore fail to reject the null hypothesis at the 5\% significance level.

Specification (5) combines specification (3) and (4) to provide a more complete picture of the effect of redlining on political outcomes. The estimated effect of redlining on political outcomes is non-linear and thus depends on the level of redlining in a census tract. Since our model is cubic in Grade D share, we notice that, holding all else equal, the relationship between Grade D share and Democrat vote share is increasing between 0 and  29.6\% Grade D share, decreasing between 29.6\% and 74.2\% Grade D share, and increasing again between 74.2\% and 100\% Grade D share. Similarly, the relationship between Grade C share and Democrat vote share is increasing between 0 and 36.1\% Grade C share, slightly decreasing between 36.1\% to 61.2\% Grade C share, and increasing again from 61.2\% Grade C share to 100\% Grade C share. We can therefore say that additional redlining is associated with an increase in Democrat vote share when redlining is low or high and a decrease in Democrat vote share when redlining is medium. These effects are small but statistically significant at the 5\% level. This is akin to a precisely estimated 0 effect, which indicates that redlining has little to no effect on modern political outcomes. It may be that redlining has had no effect on political outcomes independent of its effect on race, income, and other covariates to begin with, or it may be that the independent effect of redlining on political outcomes diminishes over time.

Specification (6) is the same as specification (5) but excludes all measures of Grade C share. Without the measures of Grade C share, our Grade D share coefficients are no longer significant at the 5\% level, which indicates that including Grade C share as a measure of redlining is justified. In Specification (5), all coefficients on Grade C share are significant and the inclusion of Grade C share leads to the coefficients on Grade D share being significant.

\section*{Limitations of Results}
While our multiple regression analysis allowed us to control for some potential sources of omitted variable bias in the form of demographic features, omitted variables still pose potential issues for the internal validity of our findings, particularly because we are concerned with long-term impacts. One particular concern comes from other geographically targeted forms of discrimination that occurred since the 1930s. For instance, blockbusting was a prevalent practice from the 1950s to 1970s in which real estate agents and developers scared white homeowners to sell their property at cheaper rates by claiming that there would soon be an influx of racial minorities, then sold those properties at higher prices to black families. This led to a mass exodus of white populations from these neighborhoods known as “white flight,” which significantly altered the racial demographics of these areas. In addition to impacting the racial makeup of these neighborhoods, this phenomenon could also impact political outcomes because the white families who were more susceptible to “white flight” would have had political views more in line with greater racial prejudice. Therefore, blockbusting and similar practices could plausibly be a source of omitted variable bias.

We also consider other potential threats to internal validity. Errors-in-variables bias could be an issue since census tracts are a relatively small unit of observation, but the reputability of our data sources and the simplicity of our covariates allow us to be relatively confident this is not an issue. Moreover, because data from South Dakota, Kentucky, and West Virginia has not been collected, there may be some element of sample selection bias. However, because this is solely a product of the states not reporting data, it is not directly tied to any of the independent or dependent variables used throughout our analysis, and we do not expect this to have any notable impact on our regression coefficients. Simultaneous causality is impossible in this study, given that the 2020 election happened nearly a century after the HOLC maps were drawn.

	Some concerns about external validity arise due to the process through which the cities in which redlining occurred were selected by the HOLC. The HOLC was only assigned to map cities with populations of 40,000 or greater as of 1930 (Aaronson et al, 2020). This means that our sample is limited to larger cities, which, as can be seen in Figure 1, are also more concentrated in the Northeast. Given that population sizes and geography can impact political outcomes, our ability to generalize our findings regarding the impact of redlining, or other forms of geographically targeted credit discrimination, on political outcomes are limited. Another limiting factor in this regard is our measure of political outcomes. Since we only use data from the 2020 election, our method does not account for changes in the effect over time, and for different federal elections. This is particularly true due to the nature of the 2020 election, which featured record voter turnouts due to expanded mail balloting, early voting programs due to COVID-19, and the contentious race between Trump and Biden (Desilver, 2021).

\section*{Conclusion}
Our research focused on finding a link between redlining and political outcomes in the context of 1930s HOLC redlining and the 2020 United States general election. After controlling for known determinants of political outcomes, we precisely estimate a near-zero effect of redlining on modern-day political outcomes. This is in contrast with some studies that show a large and persistent effect of redlining on socioeconomic outcomes. Although our study adds to the literature on redlining by examining its effects on the previously-unstudied political dimension, we are limited by omitted variables such as policy shocks and measurement error that could bias our results.


\newpage
\printbibliography
\newpage
\section*{Appendix}
\datatable{sum_stat}{sum_stat}{Summary statistics of census tracts with more than 90\% of area in HOLC maps}
\newfigure[0.8]{HOLC_cities.png}{HOLC_cities}{Locations of cities included in our study}
\newfigure[0.8]{Brooklyn_HOLC_scan_resized.png}{Brooklyn}{An example redlining map of Brooklyn, New York. Red areas indicate Grade D areas, yellow areas indicate Grade C areas, blue areas indicate Grade B areas, and green areas indicated Grade A areas.}
\datatable[footnotesize]{main_regressions}{main_regressions}{Multiple linear regression results}

\newsubfig{cshare_scatter.png}{dshare_scatter.png}{Scatterplots of Democrat vote share on Grade C percentage and Grade D percentage}{scatters}
\newsubfig{cshare_marginplot.png}{dshare_marginplot.png}{Estimated relationship between redlining (Grade D share and Grade C share) and Democrat vote share as estimated by Specification (5), Table \ref{table:main_regressions} with 95\% confidence intervals. Values of all other covariates are held constant at their mean values.}{margins}

\end{document}


